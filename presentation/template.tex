%%%%%%%%%%%%%%%%%%%%%%%%%%%%%%%%%%%%%%%%%
% Focus Beamer Presentation
% LaTeX Template
% Version 1.0 (8/8/18)
%
% This template has been downloaded from:
% http://www.LaTeXTemplates.com
%
% Original author:
% Pasquale Africa (https://github.com/elauksap/focus-beamertheme) with modifications by 
% Vel (vel@LaTeXTemplates.com)
%
% Template license:
% GNU GPL v3.0 License
%
% Important note:
% The bibliography/references need to be compiled with bibtex.
%
%%%%%%%%%%%%%%%%%%%%%%%%%%%%%%%%%%%%%%%%%

%----------------------------------------------------------------------------------------
%	PACKAGES AND OTHER DOCUMENT CONFIGURATIONS
%----------------------------------------------------------------------------------------

\documentclass{beamer}

\usetheme{focus} % Use the Focus theme supplied with the template
% Add option [numbering=none] to disable the footer progress bar
% Add option [numbering=fullbar] to show the footer progress bar as always full with a slide count

% Uncomment to enable the ice-blue theme 
%\definecolor{main}{RGB}{92, 138, 168}
%\definecolor{background}{RGB}{240, 247, 255}

%------------------------------------------------

\usepackage{booktabs} % Required for better table rules
\usepackage{tikz}
%\usepackage[uft-8]{inputenc}

%----------------------------------------------------------------------------------------
%	 TITLE SLIDE
%----------------------------------------------------------------------------------------

\title{Exploring interpretations \\ of probabilistic programs \\ using measure theory in Coq}

%\subtitle{}

\author{Kira Kutscher \& Lasse Letager Hansen}

%\institute{Aarhus University}

\date{28th of June, 2019}

%------------------------------------------------

\begin{document}
\newcommand\rml{$\mathcal{R}$\texttt{ml}} % for a pretty version of "Rml"
\maketitle

%------------------------------------------------

\begin{frame}{Motivation}
  Probabilistic algorithms are widely used and important...\\
    \pause
    ...but much less widely proven correct.\\
    \bigskip
    \pause
    We aim towards:
    \pause
    \begin{itemize}
        \item Mechanising proofs
      \pause
        \item Making proofs reliable
      \pause
        \item Making the process of correctness proofs accessible. 
    \end{itemize}
    \pause
    \bigskip
    \medskip
    Did we achieve this?
    \pause
    \begin{flushright}
      No... but we made a step towards it. 
    \end{flushright}
\end{frame}

%------------------------------------------------

\section{The shoulders of giants \\ (previous work)}

%------------------------------------------------

\begin{frame}{Two languages}
  \begin{columns}
    \column{0.5\textwidth}
    \LARGE{\rml}\tiny{\cite{rml-paper}}
    \hrule
    \column{0.5\textwidth}
    \LARGE{\texttt{pwhile}}\tiny{\cite{easy-crypt}}
    \vspace{-0.5em}
    \hrule
  \end{columns}
  \pause
  \begin{columns}
    \column{0.5\textwidth}
    \begin{itemize}
        \item functional
        \item designed as part of the ALEA library
        \item no implementation publicly available
    \end{itemize}
    \pause
    \column{0.5\textwidth}
    \begin{itemize}
        \item imperative
        \item designed as the definition language in \textsc{EasyCrypt}
        \item implemented in Coq in the \texttt{xhl} development by Pierre-Yves
      Strub
    \end{itemize}
  \end{columns}
\end{frame}

% ------------------------------------------------

\begin{frame}{\rml}
  
\begin{align*}
  exp~::=&~ ~ x~\vert ~ c~\vert ~ \texttt{if }b\texttt{ then }e_1\texttt{ else } e_2~\\
  &\vert ~ \texttt{let }x = e_1 \texttt{ in }e_2~\vert ~ f~e_1~\dots~e_n
\end{align*}
\end{frame}

% ------------------------------------------------

\begin{frame}{\texttt{pwhile}}
  
\begin{align*}
  exp ::=~& x ~\vert ~ const ~\vert ~ \texttt{prp ($p$ : pred mem)}~\vert ~ e_1\ e_2\\
  cmd ::=~& \texttt{abort} ~\vert ~ \texttt{skip} ~\vert ~ x := e ~\vert ~ x\ \$\hspace{-0.25em}= e\\
  \vert ~ & \texttt{if } b \texttt{ then } c_1 \texttt{ else } c_2 ~\vert ~
            \texttt{while } b \texttt{ do } c ~\vert ~ c_1 ; c_2
\end{align*}
\end{frame}

%----------------------------------------------------------------------------------------

\bibliographystyle{apalike}
\bibliography{sources}

\end{document}
