%%%%%%%%%%%%%%%%%%%%%%%%%%%%%%%%%%%%%%%%%
% Focus Beamer Presentation
% LaTeX Template
% Version 1.0 (8/8/18)
%
% This template has been downloaded from:
% http://www.LaTeXTemplates.com
%
% Original author:
% Pasquale Africa (https://github.com/elauksap/focus-beamertheme) with modifications by 
% Vel (vel@LaTeXTemplates.com)
%
% Template license:
% GNU GPL v3.0 License
%
% Important note:
% The bibliography/references need to be compiled with bibtex.
%
%%%%%%%%%%%%%%%%%%%%%%%%%%%%%%%%%%%%%%%%%

%----------------------------------------------------------------------------------------
%	PACKAGES AND OTHER DOCUMENT CONFIGURATIONS
%----------------------------------------------------------------------------------------

\documentclass{beamer}

\usetheme{focus} % Use the Focus theme supplied with the template
% Add option [numbering=none] to disable the footer progress bar
% Add option [numbering=fullbar] to show the footer progress bar as always full with a slide count

% Uncomment to enable the ice-blue theme 
%\definecolor{main}{RGB}{92, 138, 168}
%\definecolor{background}{RGB}{240, 247, 255}

%------------------------------------------------

\usepackage{booktabs} % Required for better table rules
\usepackage{tikz-cd}
%\usepackage[uft-8]{inputenc}

%----------------------------------------------------------------------------------------
%	 TITLE SLIDE
%----------------------------------------------------------------------------------------

\title{Exploring interpretations \\ of probabilistic programs \\ using measure theory in Coq}

%\subtitle{}

\author{Kira Kutscher \& Lasse Letager Hansen}

%\institute{Aarhus University}

\date{28th of June, 2019}

%------------------------------------------------

\begin{document}
\newcommand\rml{$\mathcal{R}$\texttt{ml}} % for a pretty version of "Rml"
\maketitle

%------------------------------------------------

\begin{frame}{This talk}
  We will...
  \pause
  \begin{itemize}
      \item be motivated by cryptography;
  \pause
      \item look at some previous work presenting the probabilistic languages \rml\
    and \texttt{pwhile};
  \pause
      \item study the theory of how to interpret probabilistic definitions as well as
    general recursion in the deterministic and certainly terminating Coq; 
  \pause
    \item see how this theory is employed in the semantics of \rml\ and \texttt{pwhile};
  \pause
    \item get a quick overview of how we used this in our project. 
  \end{itemize}
\end{frame}

%------------------------------------------------

\begin{frame}{Motivation}
  Probabilistic algorithms are widely used and important...\\
    \pause
    ...but much less widely proven correct.\\
    \bigskip
    \pause
    We aim towards:
    \pause
    \begin{itemize}
        \item Mechanising proofs
      \pause
        \item Making proofs reliable
      \pause
        \item Making the process of correctness proofs accessible. 
    \end{itemize}
    \pause
    \bigskip
    \medskip
    Did we achieve this?
    \pause
    \begin{flushright}
      No... but we made a step towards it. 
    \end{flushright}
\end{frame}

%------------------------------------------------

\section{The shoulders of giants}
%Many people. Most famously Newton

%------------------------------------------------

\begin{frame}{Two languages}
  \begin{columns}
    \column{0.5\textwidth}
    \LARGE{\rml}\tiny{\cite{rml-paper}}
    \hrule
    \column{0.5\textwidth}
    \LARGE{\texttt{pwhile}}\tiny{\cite{easy-crypt}}
    \vspace{-0.5em}
    \hrule
  \end{columns}
  \pause
  \begin{columns}
    \column{0.5\textwidth}
    \begin{itemize}
        \item functional
        \item designed as part of the ALEA library
        \item no implementation publicly available
    \end{itemize}
    \pause
    \column{0.5\textwidth}
    \begin{itemize}
        \item imperative
        \item designed as the definition language in \textsc{EasyCrypt}
        \item implemented in Coq in the \texttt{xhl} development by Pierre-Yves
      Strub
    \end{itemize}
  \end{columns}
\end{frame}

% ------------------------------------------------

\begin{frame}{\rml}
  
\begin{align*}
  exp~::=&~ ~ x~\vert ~ c~\vert ~ \texttt{if }b\texttt{ then }e_1\texttt{ else } e_2~\\
  &\vert ~ \texttt{let }x = e_1 \texttt{ in }e_2~\vert ~ f~e_1~\dots~e_n
\end{align*}

Probabilistic functions are \texttt{flip} and \texttt{random $n$}. 
\end{frame}

% ------------------------------------------------

\begin{frame}{\texttt{pwhile}}
  
\begin{align*}
  exp ::=~& x ~\vert ~ const ~\vert ~ \texttt{prp ($p$ : pred mem)}~\vert ~ e_1\ e_2\\
  cmd ::=~& \texttt{abort} ~\vert ~ \texttt{skip} ~\vert ~ x := e ~\vert ~ x\ \$\hspace{-0.25em}= e\\
  \vert ~ & \texttt{if } b \texttt{ then } c_1 \texttt{ else } c_2 ~\vert ~
            \texttt{while } b \texttt{ do } c ~\vert ~ c_1 ; c_2
\end{align*}
Probabilistic functions are \texttt{flip} and \texttt{random $n$}. 
\end{frame}

% ------------------------------------------------

\begin{frame}{Combining both approaches}
  \begin{center}
    \begin{tikzcd}
      \texttt{pwhile} \arrow[rr] \arrow[d]
      \pgfmatrixnextcell \pgfmatrixnextcell  \omega\text{-cpos} \\
      \text{\rml} \arrow[rru]    
    \end{tikzcd}
  \end{center}
\end{frame}

% ------------------------------------------------

\section{The most practical solution\\
is a [bit of] good theory}
%Albert Einstein

% ------------------------------------------------

\begin{frame}{Probabilistic definitions}
  
\end{frame}

% ------------------------------------------------

\begin{frame}{The measure monad}
  \begin{align*}
    \texttt{unit} & :~ \tau\to\texttt{M}\tau\\
                  & = \texttt{fun }(x:\tau)\Rightarrow
                    \texttt{fun }(f~:~\tau\to[0,1])\Rightarrow f~x\\
    \\
    \texttt{bind} & :~\texttt{M}\tau\to(\tau\to\texttt{M}\sigma)\to\texttt{M}\sigma\\
                  & = \texttt{fun }(\mu~:~\texttt{M}\tau)\Rightarrow \texttt{fun }
                    (g~:~\tau\to\texttt{M}\sigma) \Rightarrow\\
                  & ~~~~~\texttt{fun }(f~:~\sigma\to[0,1])\Rightarrow \mu~ (\texttt{fun
                    }(x~:~\tau)\Rightarrow M~x~f)
  \end{align*}
\end{frame}

% ------------------------------------------------

\begin{frame}{Fixpoint iterations}
  $$\textit{fac}(n) := \texttt{ if } n = 0 \texttt{ then }1\texttt{ else } n\cdot
  \textit{fac}(n-1)$$
  \pause
  $$F(g(n)):=\texttt{ if }n=0\texttt{ then }1\texttt{ else }n\cdot g(n-1)$$
  \pause
  \begin{align*}
    \hspace{0.6em}
    F_1(n) = F(F_0(n)) & = \begin{cases}
      1~~~\text{if }n\text{ is 0}\\
      0~~~\text{otherwise}
    \end{cases}
  \end{align*}
  \pause
  \vspace{-1em}
  \begin{align*}
    F_2(n) = F(F(F_0(n))) & = \begin{cases}
      1~~~\text{if }n\text{ is 0 or 1}\\
      0~~~\text{otherwise}
    \end{cases}
  \end{align*}
  \pause
  \vspace{-1em}
  \begin{align*}
    F_3(n) = F(F(F(F_0(n)))) & = \begin{cases}
      1~~~\text{if }n\text{ is 0 or 1}\\
      2~~~\text{if }n\text{ is 2}\\
      0~~~\text{otherwise}\hspace{2em}
    \end{cases}
  \end{align*}
  \pause
  \vspace{-1em}
  \begin{align*}
    \vdots \hspace{5em} \vdots \hspace{5em} \hspace{3em}\vdots\hspace{5em}
  \end{align*}
\end{frame}

% ------------------------------------------------

\begin{frame}{$\omega$-cpos}
  \begin{align*}
    X & \text{ : a set}\\
    \leq & \text{ : reflexivie, transitive, antisymmetric}\\
    0_X &\text{ : } \forall x \in X : 0 \leq x\\
    \texttt{lub}_X &~~\text{ for monotonic sequences}\\
  \end{align*}
\end{frame}

% ------------------------------------------------

\begin{frame}{Measures are $\omega$-cpos}
  \begin{align*}
    (\tau\to[0,1])\to[0,1]
  \end{align*}
  \pause
  \begin{center}
    is an $\omega$-cpo, since the definition can be extended to function spaces: 
  \end{center}
  \pause
  \begin{align*}
    f\leq_{A \to B} g \Leftrightarrow \forall x: f(x) \leq_B g(x)
    & ~~~\textit{(pointwise order)}\\
    0_{A\to B} := f(x) = 0_B
    & ~~~\textit{(least element)}\\
    \texttt{lub}_{A\to B} f_n := \texttt{lub}_B(f_n(x))
    & ~~~\textit{(least upper bound operation)}
  \end{align*}
\end{frame}

% ------------------------------------------------

\section{A language is not just words}
%Noam Chomsky

% ------------------------------------------------

\begin{frame}{Semantics of \rml}
  
\end{frame}

% ------------------------------------------------

\begin{frame}{Semantics of \texttt{pwhile}}

\end{frame}

% ------------------------------------------------

\section{Implementation beats oration}
%Aesop

% ------------------------------------------------

\begin{frame}{Our development}

\end{frame}

%----------------------------------------------------------------------------------------

\bibliographystyle{apalike}
\bibliography{sources}

\end{document}
