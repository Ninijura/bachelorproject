\documentclass{article}

% Imports
\usepackage[utf8]{inputenc}
\usepackage[margin=1in]{geometry}
\usepackage{lastpage}
\usepackage{fancyhdr}
\usepackage{graphicx}

% Header
\setlength{\headheight}{48.14pt} 
\fancyhf{}
\fancyhead[OL]{
\includegraphics[scale=0.3]{AUlogo}
\vspace{-0.5cm}
}
\fancyhead[OR]{
\begin{tabular}{l}
\textbf{\sc Bachelor Contract}       \\
Date: \today              \\
Page \thepage/\pageref{LastPage}\\
~~
\end{tabular}}

\newcommand{\timeest}[1]{$\mathbf{#1}$}% used to write time estimations without excessive math-mode

\begin{document}
\pagestyle{fancy}

% Meta-information about group/advisors/etc.
\bgroup\def\arraystretch{1.5}
\begin{table}[h]
\begin{tabular}{ll}
\textbf{Advisor}     & Bas Spitters     \\
\textbf{Students}    & Kira Kutscher    \\
\textbf{Languages}   & English          \\
\textbf{Text tools}  & \LaTeX           \\
\textbf{Other tools} & Coq              
\end{tabular}
\end{table}
\egroup\vspace{-0.cm}

\subsection*{Project Description}
The project will focus on developing a framework for cryptographic proofs in Coq. We
will investigate how much of the convenience found in tools like cryptHOL (for
Isablle) or Easycrypt we can recreate within the more powerful logic of Coq. To this
end we will first investigate existing tools (like the above mentioned) and libraries/tools
for Coq that might be of use for our development (like xpl, FCF, coqhammer, and
coqSMT).

After the right basis is selected we will start selecting sample programs and trying
to prove things about them in Coq. At this point we will decide in which direction
the project is going to continue. 


\subsection*{Provisional Table of Contents}
\begin{itemize}
    \item Abstract (10-20 lines)
    \item Section 1: Introduction (1-2 pages)
    \item Section 2: Theory and existing frameworks (4-8 pages)
    \item Section 3: Our approach (6-12 pages)
    \item Section 4: Our contribution (6-12 pages)
    \item Section 5: Comparison to other work and ideas for future work (2-4 pages)
    \item Section 6: Conclusion (1-2 pages)
    \item Acknowledgements (3-5 lines)
    \item References ($\frac{1}{2}$-1 page)
    \item Appendix with programming code, tables, full proofs, etc. (5-20 pages)
\end{itemize}

\subsection*{Provisional Time Plan}

\paragraph{First week of February (15 hours)}~\\\noindent
Planning of activities, including the production of the Bachelor's contract.

\paragraph{Rest of February and first half of March (\timeest{3\times 15} hours)}~\\\noindent
Read literature and research the current state of the art tools for formal
cryptographic proofs; rough draft of Section 2 in Bachelor's report.

\paragraph{Rest of March until start of May (\timeest{2\times 15+8\times 30} hours)}~\\\noindent
Work on the design and implementation of our contribution. Keep adding to the report
along the way. 

\paragraph{End of May and first half of June (\timeest{3\times 30} hours)}~\\\noindent
Write the missing parts, put drafts together, make things consistent, proof reading.

\end{document}
