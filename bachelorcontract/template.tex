\documentclass{article}

% Imports
\usepackage[utf8]{inputenc}
\usepackage[margin=1in]{geometry}
\usepackage{lastpage}
\usepackage[hidelinks]{hyperref}
\usepackage{fancyhdr}
\usepackage{graphicx}

% Header
\setlength{\headheight}{48.14pt} 
\fancyhf{}
\fancyhead[OL]{
\includegraphics[scale=0.3]{AUlogo}
\vspace{-0.5cm}
}
\fancyhead[OR]{
\begin{tabular}{l}
\textbf{\sc Bachelor Contract}       \\
Date: \today              \\
Page \thepage/\pageref{LastPage}\\
~~
\end{tabular}}

\newcommand{\timeest}[1]{$\mathbf{#1}$}% used to write time estimations without excessive math-mode

\begin{document}
\pagestyle{fancy}
\newcommand\rml{$\mathcal{R}$\texttt{ml} }
\newcommand\rmlno{$\mathcal{R}$\texttt{ml}}

% Meta-information about group/advisors/etc.
\bgroup\def\arraystretch{1.5}
\begin{table}[h]
\begin{tabular}{ll}
\textbf{Advisor}     & Bas Spitters     \\
\textbf{Students}    & Kira Kutscher    \\
\textbf{Languages}   & English          \\
\textbf{Text tools}  & \LaTeX           \\
\textbf{Other tools} & Coq              
\end{tabular}
\end{table}
\egroup\vspace{-0.cm}

\subsection*{Project Description}
We selected the basis for our work to consist of \rmlno, presented by Audebaud and
Paulin-Mohring in 2009 (``Proofs of randomized algorithms in Coq'') as well as the
\texttt{xhl} development\footnote{\url{https://github.com/strub/xhl}}, which is a Coq
implementation of \texttt{pwhile}, the probabilistic imperative language used in
EasyCrypt.

Our project explores similarities as well as differences between \rml and
\texttt{pwhile} and understanding their respective interpretations. Our aim for this
project is to understand how we can translate a program, $P$, written in
\texttt{pwhile} to a semantically equivalent one in \rml and show that the results of
interpreting $P$ directly will lead to the same result as interpreting the \rmlno
-translation of $P$. 

Along the way we will attempt to implement an interpretation of \rml in Coq. If we
are left with more time on our hands before the completion of this project, we will
also attempt a formalisation of the translation of \texttt{pwhile} to our
implementation of \rml. 
\\ \\
The goal of this project is to present some different approaches to the construction
and interpretation of probabilistic programs and show their equivalence. The reader
should be left with an actionable understanding of the domain theoretic approach,
enabling them to start working with the basic concepts of the topic. 

\subsection*{Provisional Table of Contents}
\begin{itemize}
    \item Abstract (10-20 lines)
    \item Section 1: Introduction (1-2 pages)
    \item Section 2: Theory and existing frameworks (6-12 pages)
    \item Section 3: Our approach (6-12 pages)
    \item Section 4: Our contribution (6-10 pages)
    \item Section 5: Comparison to other work and ideas for future work (2-4 pages)
    \item Section 6: Conclusion (1-2 pages)
    \item Acknowledgements (3-5 lines)
    \item References ($\frac{1}{2}$-1 page)
    \item Appendix with programming code, tables, full proofs, etc. (5-20 pages)
\end{itemize}

\subsection*{Provisional Time Plan}

\paragraph{First week of February (15 hours)}~\\\noindent
Planning of activities, including the production of the Bachelor's contract.

\paragraph{Rest of February and first half of March (\timeest{3\times 15} hours)}~\\\noindent
Read literature and research the current state of the art tools for formal
cryptographic proofs; rough draft of Section 2 in Bachelor's report.

\paragraph{Rest of March until start of May (\timeest{2\times 15+8\times 30} hours)}~\\\noindent
Work on the design and implementation of our contribution. Keep adding to the report
along the way. 

\paragraph{End of May and first half of June (\timeest{3\times 30} hours)}~\\\noindent
Write the missing parts, put drafts together, make things consistent, proof reading.

\paragraph{31st of May}~\\\noindent
Have a draft of the section about our own implementation. It should contain all of
the structure needed, but does not have to be clean yet.

\paragraph{3rd of June}~\\\noindent
The sections on theory as well as our own implementation should be complete modulo
proof reading.

\paragraph{10th of June}~\\\noindent
Sections 3 and 5 should be done modulo proof reading.

\paragraph{12th of June}~\\\noindent
Introduction, conclusion and abstracts (one in Danish, one in English) should be in
place. The rest of the time is spent on proof reading. 

\end{document}
